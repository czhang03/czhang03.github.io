% * Flags
%
% To declare a new flag `myflag` with the value `true`, simply add
%
%     \newif\ifmyflag\myflagtrue
%
% To prevent conflicts when editing documents collaboratively, if you want to
% temporarily change the contents of a flag, you should do so under a separate
% file `flags.tex` (preferably not kept under version control). E.g.:
%
%     \myflagtrue

\makeatletter \@input{flags} \makeatother

\documentclass{article}

% Packages

\usepackage{mathpartir}
\usepackage{unicode-math}
\usepackage{fontspec}
\usepackage{amsmath}
\usepackage{amsthm}
\usepackage[hidelinks=true]{hyperref}
\usepackage{cleveref}
\usepackage{tikz-cd}
\usepackage[backend=biber, style=authoryear-comp]{biblatex}

% Fonts
\setmainfont{Libertinus Serif}
\setmathfont{STIX Two Math}


% Theorem environment declarations

\theoremstyle{plain}
\newtheorem{theorem}{Theorem}[section]
\newtheorem{lemma}[theorem]{Lemma}
\newtheorem{proposition}[theorem]{Proposition}
\newtheorem{fact}[theorem]{Fact}
\newtheorem{claim}[theorem]{Claim}
\newtheorem{corollary}[theorem]{Corollary}
\newtheorem{convention}[theorem]{Convention}
\newtheorem{conjecture}[theorem]{Need-to-Check}

\theoremstyle{definition}
\newtheorem{definition}[theorem]{Definition}
\newtheorem{example}[theorem]{Example}
\newtheorem{counterexample}[theorem]{Counterexample}

\theoremstyle{remark}
\newtheorem{remark}[theorem]{Remark}

% Bibliography

\bibliography{refs.bib}


\title{A Functional View To Completeness of Algebra}
\author{Cheng Zhang}
\date{last modifed: \today}


\DeclareMathOperator{\KAT}{\mathsf{KAT}}
\DeclareMathOperator{\TopKAT}{\mathsf{TopKAT}}
\DeclareMathOperator{\im}{\mathrm{im}}

\begin{document}

\maketitle


\section{Introduction}

Completeness is a important property in the study of algebras. 
Formally, a class of models are complete for a algebra \(𝒜\) when 
a equality is always valid in this class of models \(M\) implies 
the equality is derivable using the algebra:
\[M ⊧ t₁ = t₂ ⇔ 𝒜 ⊢ t₁ = t₂.\]

The importance of completeness basically allows us to move between 
models and algebra whenever it is convenient:
\begin{itemize}
    \item We can reason about equality in a simple complete model, 
        and know that such a equality holds in general.
        For example, when dealing with Kleene Algebra, or Kleene algebra 
        with tests, we typically reason in language models, which is complete
        \parencite{Kozen_1994, Kozen_1997}. 
    \item When we care about equality in a particular model, 
        like relational model to model programs 
        \parencite{Kozen_1997,Smolka_Foster_Hsu_Kappé_Kozen_Silva_2020}.
        Then we know that all the equalities in the model we care about can 
        be derived just using the equational theory.
    \item Further more, we can prove or disprove 
        the equality in a complicated model in a simpler model.
\end{itemize}

Despite the importance of completeness of algebraic theory,
completeness is generally very hard to prove from scratch 
\parencite{Kozen_1994,Smolka_Foster_Hsu_Kappé_Kozen_Silva_2020}.
Hence this give birth to the idea of reduction, 
seen in \cite{Kozen_1997} and later used in many different works.
\cite{Pous_Rot_Wagemaker_2022} generalized the idea of reduction
and used to it prove several important completeness theorem 
for extensions of Kleene Algebras.


\section{The Functional Account}

It is unsurprising that the free model will be important to this development.
\begin{definition}
    A free model \(F_𝒜\) of a algebra \(𝒜\) over 
    a set of primitive \(P\) (or many sets if the algebra is multi-sorted)
    is all the terms from over \(P\) with operations in \(𝒜\) 
    mod out by provable equalities.
\end{definition} 
In other words, 
\[F_𝒜(P) ⊧ t₁ = t₂ \iff 𝒜 ⊢ t₁ = t₂,\]
by the definition of free model,
where \(t₁\) and \(t₂\) are two terms formed by primitives in \(P\).
This means that free model is a complete model, by definition.

Free model are typically also used to define a interpretation.
\begin{definition}
    A interpretation in model \(M\) of algebra \(𝒜\) is a 
    homomorphism (a function that preserves all the operation and sort of \(𝒜\))
    from the free model to \(M\). i.e. \(I: F_𝒜(P) → 𝒜\).

    A interpretation is complete when \(M ⊧ I(t₁) = I(t₂) \iff 𝒜 ⊢ t₁ = t₂\)
    for all term \(t₁\) and \(t₂\).
\end{definition}

It is straightforward to see the \(\impliedby\) direction 
for completeness condition will always hold 
because \(I\) is a homomorphism,
and the \(\implies\) direction is true 
when \[M ⊧ I(t₁) = I(t₂) \implies 𝒜 ⊢ t₁ = t₂ \implies F_𝒜(P) ⊧ t₁ = t₂,\]
which exactly describe that \(I\) is injective.
Hence:
\begin{corollary}
    a interpretation \(I\) is complete iff it is injective.
\end{corollary}

We define a extension of a algebraic theory 
\begin{definition}
    \(𝒜_{ext}\) is a extension of an algebra \(𝒜\),
    when there exists a homomorphism \([-]: F_{𝒜}(P) → F_{𝒜_{ext}}(P)\) in \(𝒜\).
\end{definition}
The map \([-]\) will simply embed a term in \(𝒜\) into \(𝒜_{ext}\).
Notice that there is no grantee that such a homomorphism is 
\begin{itemize}
    \item injective, because \(𝒜_{ext}\) can have more equations than \(𝒜\);
    \item surjective, because \(𝒜_{ext}\) can have more operations than \(𝒜\).
\end{itemize}
The only thing we know is such a map need to be a homomorphism in \(𝒜\),
since \(𝒜_{ext}\) are not allowed to remove equations and operations from \(𝒜\).

Finally, a functional definition of reduction 
\begin{definition}
    a reduction from an \(n\)-sorted extension \(𝒜_{ext}\) to 
    a \(m\)-sorted algebra \(𝒜\) is 
    \begin{itemize}
        \item a surjective map on primitives \(r: Set^m → Set^n\) 
        \item injective \(𝒜_{ext}\)-homomorphism \(r: F_{𝒜}(P) ↪ F_{𝒜_{ext}}(r(P))\)
    \end{itemize} 
\end{definition}
We are abusing notation here using \(r\) for two different map, 
since which \(r\) is used can be inferred from context.

The definition of reduction might seem obscure here, 
but it is essentially taking advantage of the fact that a 
injective interpretation is complete.
\begin{theorem}
    Given a complete interpretation \(I\) of \(𝒜\) and a reduction to 
\end{theorem}




\section{Acknowledgement}

This work is inspired by an unpublished proof by Arthur Azevedo de Amorim 
of the completeness on TopKAT with respect to language model 
and general relational models.












\printbibliography

\end{document}

% Local Variables:
% TeX-engine: luatex
% End:
